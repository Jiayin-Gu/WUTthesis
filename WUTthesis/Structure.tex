\section{模板结构}


\par \texttt{WUTthesis}的文件夹中包含以下文件及子文件夹
\begin{itemize}
\item \texttt{Thesis.tex},论文的主源文件;
\item \texttt{WUTthesis.sty},自定义的论文模板宏包;
\item \texttt{Cover.tex},包含封面信息及封面生成命令的源文件;
\item \texttt{Dedications.tex}, 包含“{\kaishu 献给某某某}”字段的源文件;
\item \texttt{Declaration.tex},包含独创性声明和学位论文使用授权书的源文件;
\item \texttt{Abstract.tex},包含中英文摘要的源文件;
\item \texttt{Introduction}等,章文件夹,多个,每一个文件夹对应一章,一章中所涉及的一切子源文件以及插图都包含在相应文件夹中;
\item \texttt{Appendices},附录文件夹,所有的附录源文件都包含在该文件夹中;
\item \texttt{Bibliography.bib},文献数据库文件;
\item \texttt{Achievements.tex},包含作者简历和科研成果的源文件;
\item \texttt{Acknowledgements.tex},包含致谢的源文件;
\item \texttt{STZhongsong.ttf},华文中宋字体文件;
\item \texttt{WUT.jpg},图片形式的“武汉理工大学”。
\end{itemize}
其中,\texttt{Dedications.tex}可有可无,不需要时,只需要将\texttt{Thesis.tex}文件中的对应导入文件的代码删除或注释。另外,该页也以可用其他软件设计生成,随后将生成的\texttt{Dedications.pdf}文档插入即可,对应的操作是将\texttt{Thesis.tex}中的对应代码替换为
\begin{lstlisting}[language=TeX]
\clearpage{\pagestyle{empty}\cleardoublepage} %% 其作用是,如果此页为偶数页,则设为完全空白页,进入下一页
\includepdf{Dedications.pdf} %% 导入Dedications.pdf文档
\end{lstlisting}
即可。关于{\STZhongsong 独创性声明}和{\STZhongsong 学位论文使用授权书}部分,可以打印、签名、扫描后将成生的\texttt{Declaration.pdf}文档插入;对应的操作是将\texttt{Thesis.tex}中的对应导入文件的代码替换为
\begin{lstlisting}[language=TeX]
\clearpage{\pagestyle{empty}\cleardoublepage}
\includepdf{Declaration.pdf}
\end{lstlisting}
即可。每一章的层次结构分为章(chapter)、节(section)、小节(subsection),这里建议每一节的所有内容都写进一个源文件中,然后使用导入命令 \verb"\input{}" 将各个源文件以类似递归的方式链接成一个整体(\texttt{Thesis.tex}中导入对应各章的\texttt{Chapter.tex},各章的\texttt{Chapter.tex}中在导入对应的各节的源文件)。每一章所涉及的所有内容(每一节对应的源文件、图)都存放到一个文件夹,这样做的目的也是为了使整个\texttt{WUTthesis}的结构更加突出,便于用户理解。然而,这样做就面临一个问题,当导入对应节(section)的源文件或图是,都必须提供精准的路径,而当对应每一章的文件夹重命名后,相应章的所有路径都必须改变,这因此会造成一些麻烦。解决的办法是在\texttt{Thesis.tex}中导入每一章源文件前定义一个指代路径的宏 \verb"\path",例如:
\begin{lstlisting}[language=TeX]
\def\path{Introduction}\chapter{\texttt{WUTthesis}详解}\label{chap_WUTthesis}




\section{模板结构}


\par \texttt{WUTthesis}的文件夹中包含以下文件及子文件夹
\begin{itemize}
\item \texttt{Thesis.tex},论文的主源文件;
\item \texttt{WUTthesis.sty},自定义的论文模板宏包;
\item \texttt{Cover.tex},包含封面信息及封面生成命令的源文件;
\item \texttt{Dedications.tex}, 包含“{\kaishu 献给某某某}”字段的源文件;
\item \texttt{Declaration.tex},包含独创性声明和学位论文使用授权书的源文件;
\item \texttt{Abstract.tex},包含中英文摘要的源文件;
\item \texttt{Introduction}等,章文件夹,多个,每一个文件夹对应一章,一章中所涉及的一切子源文件以及插图都包含在相应文件夹中;
\item \texttt{Appendices},附录文件夹,所有的附录源文件都包含在该文件夹中;
\item \texttt{Bibliography.bib},文献数据库文件;
\item \texttt{Achievements.tex},包含作者简历和科研成果的源文件;
\item \texttt{Acknowledgements.tex},包含致谢的源文件;
\item \texttt{STZhongsong.ttf},华文中宋字体文件;
\item \texttt{WUT.jpg},图片形式的“武汉理工大学”。
\end{itemize}
其中,\texttt{Dedications.tex}可有可无,不需要时,只需要将\texttt{Thesis.tex}文件中的对应导入文件的代码删除或注释。另外,该页也以可用其他软件设计生成,随后将生成的\texttt{Dedications.pdf}文档插入即可,对应的操作是将\texttt{Thesis.tex}中的对应代码替换为
\begin{lstlisting}[language=TeX]
\clearpage{\pagestyle{empty}\cleardoublepage} %% 其作用是,如果此页为偶数页,则设为完全空白页,进入下一页
\includepdf{Dedications.pdf} %% 导入Dedications.pdf文档
\end{lstlisting}
即可。关于{\STZhongsong 独创性声明}和{\STZhongsong 学位论文使用授权书}部分,可以打印、签名、扫描后将成生的\texttt{Declaration.pdf}文档插入;对应的操作是将\texttt{Thesis.tex}中的对应导入文件的代码替换为
\begin{lstlisting}[language=TeX]
\clearpage{\pagestyle{empty}\cleardoublepage}
\includepdf{Declaration.pdf}
\end{lstlisting}
即可。每一章的层次结构分为章(chapter)、节(section)、小节(subsection),这里建议每一节的所有内容都写进一个源文件中,然后使用导入命令 \verb"\input{}" 将各个源文件以类似递归的方式链接成一个整体(\texttt{Thesis.tex}中导入对应各章的\texttt{Chapter.tex},各章的\texttt{Chapter.tex}中在导入对应的各节的源文件)。每一章所涉及的所有内容(每一节对应的源文件、图)都存放到一个文件夹,这样做的目的也是为了使整个\texttt{WUTthesis}的结构更加突出,便于用户理解。然而,这样做就面临一个问题,当导入对应节(section)的源文件或图是,都必须提供精准的路径,而当对应每一章的文件夹重命名后,相应章的所有路径都必须改变,这因此会造成一些麻烦。解决的办法是在\texttt{Thesis.tex}中导入每一章源文件前定义一个指代路径的宏 \verb"\path",例如:
\begin{lstlisting}[language=TeX]
\def\path{Introduction}\chapter{\texttt{WUTthesis}详解}\label{chap_WUTthesis}




\input{\path/Structure.tex}
\input{\path/CTeX.tex}
\input{\path/WUTthesis_package.tex}
\input{\path/Bibliography.tex}
\input{\path/Font.tex}
\input{\path/Setting.tex}
\input{\path/Compiling.tex}




\end{lstlisting}
之后每当我们导入源文件或图时,提供 \verb"\path" 作为路径即可,重命名每一章文件夹时,只要更改对应路径的宏定义就行。通过一些设置,编译生成pdf文档中每一章的开启都位于奇数页,且开启页隐藏页眉也页码,相应的,结束页的页码为偶数,如果此时恰好结束页空白,则该页将隐藏页眉和页码。关于页码,从摘要到第一章开始前,使用罗马数字,而从第一章开始,重新编号,使用阿拉伯数字。在\texttt{Thesis.tex}中关于目录的代码如下:
\begin{lstlisting}[language=TeX]
\begin{spacing}{1.3}
\tableofcontents
\end{spacing}
\end{lstlisting}
这里的\texttt{spacing}环境由\texttt{setspace}宏包提供,其参数用于控制行距(在参数$1.3$的情况下,使用\texttt{spacing}环境和不使用\texttt{spacing}环境,排版效果是一样的)。用户可以根据实际情况,适当调整目录的行距,使得目录尽量控制在两页之内。这里需要说明一下,由于汉字都是方块字,这和英文具有大小写的情况不同,所以使用{\CTeX}宏集提供的\texttt{ctexbook}文类时,行距会自动调整为英文情形下的$1.3$倍。\texttt{Thesis.tex}中对于文献打印部分的代码
\begin{lstlisting}[language=TeX]
\begin{spacing}{1.3}
\printbibliography[heading=bibintoc, title={参考文献}]
\end{spacing}
\end{lstlisting}
是同样的道理,用户可以根据实际情况,调整参考文献的行距。{\color{red}关于书脊部分信息,可以在打印论文封面时让打印店老板操作一下,一般情况下他们都会知道怎么做。}



\section{{\CTeX}宏集}


\par {\CTeX}宏集是面向中文排版的通用{\LaTeX}排版框架,为中文{\LaTeX}文档提供了汉字输出支持、标点压缩、字体字号命令、标题文字汉化、中文版式调整、数字日期转换等支持功能,可适应论文、报告、幻灯片等不同类型的中文文档。\texttt{WUTthesis}正式基于{\CTeX}宏集提供的\texttt{ctexbook}文类制作而成,因此,关于{\CTeX}宏集的一切命令都可以\texttt{WUTthesis}。\texttt{ctexbook}文类的使用如下:
\begin{lstlisting}[language=TeX]
\documentclass[a4paper, UTF8, zihao=-4]{ctexbook}
\end{lstlisting}
其中的参数分别制定A4纸、源文件为UTF8编码方式,正文默认字号小四。{\color{red}这里需要特别强调的是,整个论文的源文件中的所有字符必须是用UTF8编码,这个非常重要!!}


\begin{color}{red}
\par 一般情况下,{\CTeX}宏集可以自动在一段文字中的中英文之间插如一段空白,以兼容中英文的排版风格。但是在有些情况下,当一段文字中的英文包含在某些命令中时,{\CTeX}宏集就无法做到自动插入空白,这时就有必要人工的方式在源文件中插入空白,例如:
\begin{lstlisting}
可从 \href{https://ctan.org/?lang=en}{CTAN} 下载
\end{lstlisting}
上述这段文字中的命令为“CTAN”引入网页链接。这些命令除了 \verb"\href{}{}" (引入网页链接),还包括 \verb"\verb""" (抄录)、\verb"\ref{}" (图、表、公式的引用)等。
\end{color}

{\color{red}\par 另外,还需要指出的是,有可能由于中文字库不完备的原因,而使得在最终生成的pdf文档中无法显示一些生辟字。(针对这一潜在的问题,还需要进一步的解决方案!)}


\par {\CTeX}的详细介绍,请参考其宏包的说明文档,可从 \href{https://ctan.org/?lang=en}{CTAN} 下载({\color{red} 强烈推荐阅读})。



\subsection{中文字体}


\par {\CTeX}宏集提供了如下几种常见的中文字体及其相应的声明形式的生成命令\footnote{针对不同操作系统所包含的不同的字库,{\CTeX}套件还可能提供了一些其他字体,比如隶书、幼圆,具体可参见{\CTeX}宏集手册。}:
\begin{itemize}
\item {\songti 宋体},\verb"\songti";
\item {\heiti 黑体},\verb"\heiti";
\item {\fangsong 仿宋},\verb"\fangsong";
\item {\kaishu 楷书},\verb"\kaishu"。
\end{itemize}
对于大段落中文文本字体的修改,我们还可以使用字体名作为环境名的环境作用形式,例如
\begin{lstlisting}[language=TeX]
\begin{heiti}
黑体文本
\end{heiti}
\end{lstlisting}
本文中,宋体是常规默认字体,而黑体则用作各级标题的默认字体。除了以上列出的四种常用字体,\texttt{WUTthesis.sty}还定义了华文中宋字体,其命令为 \verb"\STZhongsong"({\heiti 这里没有定义环境形式})。其定义是通过\texttt{xeCJK}\footnote{当使用{\CTeX}宏集时,\texttt{xeCJK}宏包会自动加载,所以无需另外加载。}宏包提供的相关命令从文件夹中所包含的华文中宋字体文件\texttt{STZhongsong.ttf}中直接提字。{\STZhongsong 独创性声明}和{\STZhongsong 学位论文使用授权书}部分就使用的华文中宋。宋体和{\heiti 黑体}还可以通过 \verb"\textbf{}"命令(参数形式)或其等价 \verb"\bfseries" 命令(声明形式)获得\textbf{加粗宋体}和\textbf{\heiti 加粗黑体}。









\subsection{字体尺寸(字号)}

\par 字体大小(字号)的设置通过{\CTeX}宏集提供的命令 \verb"\zihao{<代码>}",各字号对应的 \verb"<代码>" 及相应大小\footnote{这里的字体大小采用的是pt (point) 作为单位,$1 {\rm pt}\approx 0.35 {\rm mm}$。}可参见表~\ref{tab_zihao}。另外,对于大段文字也可采用环境形式,例如
\begin{lstlisting}[language=TeX]
\begin{zihao}{0}
武汉理工大学
\end{zihao}
\end{lstlisting}
将“武汉理工大学”设置成初号字体。因为\texttt{WUTthesis}使用了{\CTeX}宏集提供的\texttt{ctexbook}文类,{\LaTeX}的标准字体尺寸命令被重新定义,使得这些命令与中文字号有所对应,具体的对应方式见表~\ref{tab_standard_fontsize}。


\begin{table}
\caption{中文字号及相应的代码和大小。}
\begin{center}
\begin{tabular}{>{\centering\arraybackslash}m{2.0cm}|>{\centering\arraybackslash}m{2.0cm}|>{\centering\arraybackslash}m{2.0cm}||>{\centering\arraybackslash}m{2.0cm}|>{\centering\arraybackslash}m{2.0cm}|>{\centering\arraybackslash}m{2.0cm}}
\hline
\hline
字号 & 代码 & 大小 & 字号 & 代码 & 大小 \bigstrut \\ \hline
初号 & 0 & 42.15749pt & 小初 & -0 & 36.135pt \bigstrut \\ \hline
一号 & 1 & 26.09749pt & 小一 & -1 & 24.09pt \bigstrut \\ \hline
二号 & 2 & 22.08249pt & 小二 & -2 & 18.06749pt \bigstrut \\ \hline
三号 & 3 & 16.06pt & 小三 & -3 & 15.05624pt \bigstrut \\ \hline
四号 & 4 & 14.05249pt & 小四 & -4 & 12.045pt \bigstrut \\ \hline
五号 & 5 & 10.53937pt & 小五 & -5 & 9.03374pt \bigstrut \\ \hline
六号 & 6 & 7.52812pt & 小六 & -6 & 6.52437pt \bigstrut \\ \hline
七号 & 7 & 5.52061pt & 八号 & 8 & 5.01874pt \bigstrut \\ \hline
\hline
\end{tabular}
\end{center}
\label{tab_zihao}
\end{table}





\begin{table}
\caption{标准字体尺寸命令与中文字号在\texttt{ctexbook}文类选项为\texttt{zihao=-4}和\texttt{zihao=5}两种情况下的对应方式。}
\begin{center}
\begin{tabular}{>{\centering\arraybackslash}m{4.0cm}||>{\centering\arraybackslash}m{3.0cm}|>{\centering\arraybackslash}m{3.0cm}}
\hline
\hline
standard fontsize & \texttt{zihao=5} & \texttt{zihao=-4} \bigstrut \\ \hline
\verb"\tiny" & 七号 & 小六 \bigstrut \\ \hline
\verb"\scriptsize" & 小六 & 六号 \bigstrut \\ \hline
\verb"\footnotesize" & 六号 & 小五 \bigstrut \\ \hline
\verb"\small" & 小五 & 五号 \bigstrut \\ \hline
\verb"\normalsize" & 五号 & 小四 \bigstrut \\ \hline
\verb"\large" & 小四 & 小三 \bigstrut \\ \hline
\verb"\Large" & 小三 & 小二 \bigstrut \\ \hline
\verb"\LARGE" & 小二 & 二号 \bigstrut \\ \hline
\verb"\huge" & 二号 & 小一 \bigstrut \\ \hline
\verb"\Huge" & 一号 & 一号 \bigstrut \\ \hline
\hline
\end{tabular}
\end{center}
\label{tab_standard_fontsize}
\end{table}




\section{\texttt{WUTthesis}宏包}

\par 一般而言,\texttt{WUTthesis}指代的是整个模板,然而为了方便,也将包含在整个模板中的宏包文件设置为同名的\texttt{WUTthesis.sty}。当导入\texttt{WUTthesis}宏包时需要提供如下参数:
\begin{itemize}
\item Chinesetype,学位类别的中文名称,默认理学;
\item Englishtype,学位类别的英文名称,默认Science;
\item Chinesedegree,学位级别的中文名称,默认博士;
\item Englishdegree,学位级别的英文名称,默认Doctor。
\end{itemize}
\texttt{WUTthesis.sty}宏包文件中包含如下代码
\begin{lstlisting}[language=TeX]
\RequirePackage[text={160mm, 230mm}, left=28mm, vmarginratio=1:1]{geometry}
\RequirePackage{fancyhdr}
\RequirePackage{xkeyval}
\RequirePackage{changepage}
\end{lstlisting}
这表示\texttt{geometry}、\texttt{fancyhdr}、\texttt{xkeyval}、\texttt{changepage}四个宏包会被加载,因此当在\texttt{Thesis.tex}中无需再加载。其中,宏包\texttt{geometry}的参数设置了版心的大小为$160mm\times 230mm$,距离A4指左边距离$28mm$(意思是靠书脊一侧空白为$28mm$),上下空白比例$1:1$。这里有必要提一下A4纸张的大小为$210mm\times 297mm$,根据设定的版心大小和位置,内测的空白为$28mm$,靠外的空白为$22mm$,这样的不对称是为了补偿论文打印胶装后由于书脊的存在而造成整个版面内测的损失。然而,对于第一页的封面,我们又通过\texttt{changepage}宏包提供的\texttt{adjustwidth}环境调整了版心位置,使其居中。自制的\texttt{WUTthesis}中对页眉进行了设置,使得相应文字为灰色、宋体、字号小五,还将页码设置为每页底部居中。另外,还通过{\STZhongsong 华文中宋}字体文件\texttt{STZhongsong.ttf}定义了相应的字体。这里,鼓励用户查看\texttt{WUTthesis.sty},了解具体细节。







\subsection{自定义命令}

\par \texttt{WUTthesis}宏包中的自定义命令包括:
\begin{itemize}
\item \verb"\WUTclassificationnumber{}",分类号;
\item \verb"\WUTconfidentiality{}",密级:只有涉密论文才填写;
\item \verb"\WUTUDC{}",UDC;
\item \verb"\WUTuniversitycode{10497}",学校代码,参数为10497;
\item \verb"\WUTChinesetitle{}",论文中文题目;
\item \verb"\WUTEnglishtitle{}{}",论文英文题目,由于一般英文题目过长,这里分成两行,所以这里相应地设置两个参数;
\item \verb"\WUTauthor{}{}",论文作者:中文姓名、英文姓名;
\item \verb"\WUTsupervisor{}{}{}{}{}{}",指导教师:中文姓名、英文姓名、职称、学位、单位名称、邮编;
\item \verb"\WUTvicesupervisor{}{}{}{}{}{}",副指导教师:开关(只有on和off两个选项, 表明是否有副导师,如没有则在封面中不显示相关字段)、中文姓名、职称、学位、单位名称、邮编;
\item \verb"\WUTmajor{}{}",二级学科:中文专业名称、英文专业名称;
\item \verb"\WUTinstitute{}",院系名称;
\item \verb"\WUTcommitteechairman{}",答辩委员会主席;
\item \verb"\WUTreviewers{}{}",两个评阅人;
\item \verb"\WUTdegreeorganization{}",学位授予单位,即“武汉理工大学”;
\item \verb"\WUTdates{}{}{}{}{}",论文完成日期、论文提交日期、论文答辩日期、学位授予日期(格式为:xxxx年xx月)、英文日期(格式为: May, 2020);
\item \verb"\WUTdegreeabbreviation{}",学位级别类型缩写,如Ph.D.,M.S.等;
\item \verb"\WUTChinesekeywords{}",中文关键字;
\item \verb"\WUTEnglishkeywords{}",英文关键字,字体会自动设置为{\fontfamily{ptm}\selectfont Times New Roman};
\item \verb"\STZhongsong",声明形式的华文中宋字体设置。
\item \verb"\WUTmakefirstcover",生成封面一 (包含其背面的英文封面);
\item \verb"\WUTmakesecondcover",生成封面二;
\end{itemize}





\subsection{自定义环境}


\par \texttt{WUTthesis}宏包中自定义的一些环境包括:
\begin{itemize}
\item \texttt{WUTChineseabstract},中文摘要环境;
\item \texttt{WUTEnglishabstract},英文摘要环境,字体会自动设置为{\fontfamily{ptm}\selectfont Times New Roman};
\item \texttt{WUTacknowledgements},致谢环境,字体会自动设置为{\kaishu 楷书}。
\item \texttt{WUTquote},引述环境,该环境必须提供一个参数,指明引述内容的出处。
\end{itemize}





\section{参考文献}



\subsection{文献数据库文件}
\par 参考文献的信息都记录在\texttt{Bibliograph.bib}文件当中,\texttt{Thesis.tex}中的相应代码为
\begin{lstlisting}[language=TeX]
\addbibresource{Bibliography.bib}
\end{lstlisting}
文献数据库中主要分为期刊文献(\verb"@article{}")和书籍(\verb"@book{}"),比如如下文献信息:
\begin{lstlisting}
@article{Feynman_RevModPhys_1948, author={R. P. Feynman}, title={Space-time Approach to Non-relativistic Quantum Mechanics}, journal={Reviews of Modern Physics}, year={1948}, volume={20}, pages={367--387}}

@book{Hu_2013, author={胡伟}, title={{\LaTeXe} 完全学习手册}, edition={第二版}, publisher={清华大学出版社}, address={北京}, year={2013}}

@book{Knuth_1986, author={Donald E. Knuth}, title={Computers \& Typesetting, Volume A: The TeXbook}, publisher={Addison-Wesley}, year={1986}}

@book{Mittelbach_2004, author={Frank Mittelbach and Michel Goossens and Johannes Braams and David Carlisle and Chris Rowley}, title={The {\LaTeX} Companion (Tools and Techniques for Computer Typesetting)}, edition={Second}, publisher={Addison-Wesley}, year={2004}}
\end{lstlisting}
文献信息条目最好按照字母顺序排列。其中的\texttt{Hu\_2013},\texttt{Feynman\_RevModPhys\_1948}等是引用是所用的“健”。这里的建议是,对于书籍,采用作者名加年份的方式命名,对于期刊文献,采用作者名加期刊名加年份的方式命名,其中期刊名由ISO 4缩写得来,例如\texttt{RevModPhys}代表缩写\texttt{Rev. Mod. Phys.},完整的期刊名是\textit{Reviews of Modern Physics}。说到这里,如果用户对量子力学的路经积分表述感兴趣,可参考文献\cite{Feynman_RevModPhys_1948}。



\subsection{\texttt{biblatex}宏包和\texttt{gb7714-2015}样式}
\par \texttt{Thesis.tex}中导言区加载的\texttt{biblatex}负责参考文献处理的宏包,宏包导入代码如下:
\begin{lstlisting}[language=TeX]
\usepackage[backend=biber, maxbibnames=3, minbibnames=3, style=gb7714-2015, gbpub=false, gbnamefmt=lowercase]{biblatex}
\end{lstlisting}
其中的参数说明如下
\begin{itemize}
\item \texttt{biber},后端的引擎程序;
\item \texttt{maxbibnames=3},the maximum number of authors displayed in bibliography;
\item \texttt{minbibnames=3},the minimum number of authors displayed in bibliography;
\item \texttt{style=gb7714-2015},采用\texttt{gb7714-2015}样式,该样式由宏包 \\ \texttt{biblatex-gb7714-2015}提供,是胡振震根据《GB/T 7714-2015 信息与文献参考文献著录规则》的要求开发而成;
\item \texttt{gbpub=false};宏包\texttt{biblatex-gb7714-2015}提供的参数,意思是当文献出版信息缺失时,不惨用[出版地不详]、[出版者不详]等填补缺省信息,而使用标准样式的方式取消相应项的输出;
\item \texttt{gbnamefmt=lowercase},宏包\texttt{biblatex-gb7714-2015}提供的参数,意思是参考文献的作者姓名的大小写由输入信息确定不做改变。
\end{itemize}



关于\texttt{biblatex}宏包和\texttt{biblatex-gb7714-2015}宏包的详细说明,可以从 \href{https://ctan.org/?lang=en}{CTAN} 网站上下载相应的说明文档



\section{外国文字字体}

\par 由于我们对源文件采用了UTF8的编码方式,所以在源文件中可以直接输入各国的文字。然而,为了能够使得各国文字能够在最终生成的pdf文档中显示,一般情况下需要在源文件对外国文字指定相应的字体。中、日、韩文字同属东北亚文字,排版方式相近,所以可一通过\texttt{xeCJK}宏包可以定义一些相关字体(安装或直接子字体文件的形式放置于\texttt{WUTthesis}文件夹下),而\texttt{xeCJK}宏包在使用{\CTeX}宏集提供的\texttt{ctexbook}文类时已经自动导入了。下面列出一些在\texttt{Thesis.tex}导言区添加了日、韩、俄三种文字的字体设置:
\begin{itemize}
\item \verb"\setCJKfamilyfont{IPAMincho}{IPAMincho}",设置日文字体;
\item \verb"\setCJKfamilyfont{IPAGothic}{IPAGothic}",设置日文字体;
\item \verb"\setCJKfamilyfont{UnGungseo}{UnGungseo.ttf}",设置韩文字体;
\item \verb"\setCJKfamilyfont{gulim}{gulim.ttf}",设置韩文字体;
\item \verb"\newfontfamily\russian{DejaVu Serif}",设置俄文字体;
\end{itemize}
上述的字体系统均已经安装。更多字体的设置,就需要用户发挥主观能动性查找一些网络资料了。


\subsection{英文字体(西欧文字)}
\par 英文字体分为三类,分别是罗马体字族(\textrm{Roman Family})、等宽体字族(\textsf{San Serif Family})、等线体字族(\texttt{Typewriter Family})\footnote{有些书籍或相关资料中也将这三类字体称为衬线字族 (Serif)、非衬线字族 (Sans Serif)、等宽字群 (Monospace)。}。本文所使用的对应的三类字族具体为Latin Modern Roman,\textsf{Latin Modern Sans Serif}和\texttt{Latin Modern Sans Typerwriter}\footnote{Latin Modern系列字体是Computer Modern系列字体的加强版本,后者是早期Donald E. Knuth在开发{\TeX}排版系统时所开发的系列字体。},其中Latin Modern Roman为常规默认的英文字体。表 \ref{tab_font_style} 中列出了一些字体设置命令以及相应的样式。一些字体设置命令的声明形式有它们对应的简化形式,分别是:\verb"\rm" 等价于 \verb"\rmfamily",\verb"\sf" 等价于 \verb"\sffamily",\verb"\tt" 等价于 \verb"\ttfamily",\verb"\bf" 等价于 \verb"\bfseries",\verb"\it" 等价于 \verb"\itshape",\verb"\sc" 等价于 \verb"\scshape",\verb"\sl" 等价于 \verb"\slshape"。这些等价形式的命令在有些复合字体设置情况下和原有形式其实并不能做到完全意义上的等价,所以使用须谨慎。和中文字体设置一样,一些英文字体设置命令可以作为环境名,组成字体设置环境,例如加粗环境
\begin{lstlisting}[language=TeX]
\begin{bfseries}
这是字体加粗环境
\end{bfseries}
\end{lstlisting}




\begin{table}
\caption{英文字体设置命令及样式。}
\begin{center}
\begin{tabular}{>{\centering\arraybackslash}m{3.0cm}|>{\centering\arraybackslash}m{2.5cm}|>{\centering\arraybackslash}m{4.0cm}|>{\centering\arraybackslash}m{3.0cm}}
\hline
\hline
参数形式 & 声明形式 & 字样 & 说明 \bigstrut \\ \hline
\verb"\textrm{}" & \verb"\rmfamily" & \textrm{Roman Family} & 罗马体字族 \bigstrut \\ \hline
\verb"\textsf{}" & \verb"\sffamily" & \textsf{San Serif Family} & 等线体字族 \bigstrut \\ \hline
\verb"\texttt{}" & \verb"\ttfamily" & \texttt{Typewriter Family} & 罗宽体字族 \bigstrut \\ \hline
\verb"\textbf{}" & \verb"\bfseries" & \textbf{Boldface Series} & 粗宽序列 \bigstrut \\ \hline
\verb"\textmd{}" & \verb"\mdseries" & \textmd{Medium Series} & 常规序列 \bigstrut \\ \hline
\verb"\textit{}" & \verb"\itshape"  & \textit{Italic Shape} & 斜体形状 \bigstrut \\ \hline
\verb"\textsc{}" & \verb"\scshape"  & \textsc{Small Caps Shape} & 小型大写形状 \bigstrut \\ \hline
\verb"\textsl{}" & \verb"\slshape"  & \textsl{Slanted Shape} & 倾斜形状 \bigstrut \\ \hline
\verb"\textup{}" & \verb"\upshape"  & \textup{Upright Shape} & 直立形状 \bigstrut \\ \hline
\verb"\textnormal{}" & \verb"\normalfont" & \textnormal{Normal Style} & 常规字体 \bigstrut \\ \hline
\verb"\emph{}" & \verb"\em" & \emph{emphasized text} & 强调某段文字 \bigstrut \\ \hline
\hline
\end{tabular}
\end{center}
\label{tab_font_style}
\end{table}

\par {\CTeX}宏集中还在中文字体和三类英文字族之间建立了对应关系,宋体对应罗马体字族,黑体对应等线体字族,仿宋对应等宽体字族,即 \verb"\textsf{}" 作用到中文汉字相当于设置黑体,\verb"\texttt{}" 作用到中文汉字相当于设置仿宋。此外,{\CTeX}宏集还将楷书对应到英文斜体形状,即 \verb"\textit{}" 作用到中文汉字相当于设置楷书。这里,我们有必要介绍如何设置{\fontfamily{ptm}\selectfont Times New Roman}字体,这是另一种常见的英文字体,属于罗马体字族。该字体设置方法为
\begin{lstlisting}[language=TeX]
{\fontfamily{ptm}\selectfont text}
\end{lstlisting}
其中,\texttt{ptm}为对应的字体码。\texttt{WUTthesis}的英文摘要环境中已经自动设置了{\fontfamily{ptm}\selectfont Times New Roman}字体。根据各种字体的视觉特点,通常论文的正文使用罗马体字族,专有名词或程序命令选用等宽体字族。表 \ref{tab_more_English_fonts} 列出了更多的英文字体。



\subsection{日文字体}
\begin{itemize}
\item \verb"{\CJKfamily{IPAMincho} 日文}",{\CJKfamily{IPAMincho} 山川の異域は,風月天と同じである。}
\item \verb"{\CJKfamily{IPAGothic} 日文}",{\CJKfamily{IPAGothic} 山川の異域は,風月天と同じである。}
\end{itemize}





\subsection{韩文字体}

\begin{itemize}
\item \verb"{\CJKfamily{UnGungseo} 韩文}",{\CJKfamily{UnGungseo} 클래식}。
\item \verb"{\CJKfamily{gulim} 韩文}",{\CJKfamily{gulim} 클래식}。
\end{itemize}

\subsection{俄文字体}

\begin{itemize}
\item \verb"{\russian 俄文}"
\end{itemize}
\begin{center}
\russian 
{\large Я вас любил ---А.С. Пушкин} \\
\vspace{0.5cm}
Я вас любил: любовь еще, быть может, \\
В душе моей угасла не совсем; \\
Но пусть она вас больше не тревожит; \\
Я не хочу печалить вас ничем. \\
Я вас любил безмолвно, безнадежно, \\
То робостью, то ревностью томим; \\
Я вас любил так искренно, так нежно, \\
Как дай вам бог любимой быть другим. \\
\end{center}




\section{一些设置说明}

\par \texttt{WUTthesis}的导言区一些格式设置,现说明如下:
\begin{itemize}
\item \verb"\renewcommand{\theequation}{\thechapter-\arabic{equation}}",将公式代号中默认的“.”改为“-”;
\item \verb"\renewcommand{\thefigure}{\thechapter-\arabic{figure}}",将插图代号中默认的“.”改为“-”;
\item \verb"\renewcommand{\thetable}{\thechapter-\arabic{table}}",将表格代号中默认的“.”改为“-”;
\item \verb"\ctexset{chapter={format+={\zihao{-2}\heiti}," \\ \verb"number={\arabic{chapter}}, afterskip={33pt}}}",设置章标题为字号小二,黑体,阿拉伯数字,章节标题与后面下方之间的距离为33pt;
\item \verb"\ctexset{section={format+={\zihao{3}\heiti}, afterskip={21pt}}}",设置节标题为字号三号,黑体,阿拉伯数字,章节标题与后面下方之间的距离为22pt;
\item \verb"\ctexset{subsection={format+={\zihao{4}\heiti}, afterskip={7pt}}}",设置小节标题为字号四号,黑体,阿拉伯数字,章节标题与后面下方之间的距离为7pt;
\item \verb"\renewcommand{\bibfont}{\zihao{5}}",设置参考文献字号五号;
\item \verb"\renewcommand{\bibauthorfont}{\bfseries\color{teal}}",设置参考文献中作者字段为粗体,蓝绿色;
\item \verb"\renewcommand{\bibtitlefont}{\color{blue}}",设置参考文献中名称字段为蓝色;
\item \verb"\renewcommand{\bibpubfont}{\itshape\color{violet}}",设置参考文献中出版项字段为斜体,紫色。
\end{itemize}




\section{编译方式}

\par {\LaTeX}的编译方式有多种,其中具体的区别可参考相关资料,这里推荐使用\texttt{xelatex},该编译方式可以直接处理UTF8编码的字符。因为需要基于\texttt{biblatex}生成参考文献,完整的编译分为如下四步:
\begin{itemize}
\item \texttt{xelatex Thesis.tex}
\item \texttt{biber Thesis}
\item \texttt{xelatex Thesis.tex}
\item \texttt{xelatex Thesis.tex}
\end{itemize}
不过在具体的论文撰写过程中,当暂时不关心参考文献的生成时,可以仅用\texttt{xelatex Thesis.tex}编译文档来查看排版效果。{\color{red} 有时编译后发现论文中的公式或图表的链接部分显示“??”,这时只需要再编译一次即可。编译时,可能由于字体大小替换而出现一些警告,这个时候可以忽略,不影响最终编译。}









\end{lstlisting}
之后每当我们导入源文件或图时,提供 \verb"\path" 作为路径即可,重命名每一章文件夹时,只要更改对应路径的宏定义就行。通过一些设置,编译生成pdf文档中每一章的开启都位于奇数页,且开启页隐藏页眉也页码,相应的,结束页的页码为偶数,如果此时恰好结束页空白,则该页将隐藏页眉和页码。关于页码,从摘要到第一章开始前,使用罗马数字,而从第一章开始,重新编号,使用阿拉伯数字。在\texttt{Thesis.tex}中关于目录的代码如下:
\begin{lstlisting}[language=TeX]
\begin{spacing}{1.3}
\tableofcontents
\end{spacing}
\end{lstlisting}
这里的\texttt{spacing}环境由\texttt{setspace}宏包提供,其参数用于控制行距(在参数$1.3$的情况下,使用\texttt{spacing}环境和不使用\texttt{spacing}环境,排版效果是一样的)。用户可以根据实际情况,适当调整目录的行距,使得目录尽量控制在两页之内。这里需要说明一下,由于汉字都是方块字,这和英文具有大小写的情况不同,所以使用{\CTeX}宏集提供的\texttt{ctexbook}文类时,行距会自动调整为英文情形下的$1.3$倍。\texttt{Thesis.tex}中对于文献打印部分的代码
\begin{lstlisting}[language=TeX]
\begin{spacing}{1.3}
\printbibliography[heading=bibintoc, title={参考文献}]
\end{spacing}
\end{lstlisting}
是同样的道理,用户可以根据实际情况,调整参考文献的行距。{\color{red}关于书脊部分信息,可以在打印论文封面时让打印店老板操作一下,一般情况下他们都会知道怎么做。}


